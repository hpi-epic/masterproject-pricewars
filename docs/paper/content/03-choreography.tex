%
\section{Service Choreography}
\label{sec:Choreography}
% Owner: Seb
% Reviewed: jani
%
Like previously discussed, the architecture consists out of several services communicating with each other. This communication is done via well-defined RESTful APIs using JSON objects\footnote{\url{https://hpi-epic.github.io/masterproject-pricewars/}}.  Due to the micro-service architecture and the design goal to allow real competition of different merchants without cheating, all important routes are secured by authorization tokens. For authenticating all participants in the simulation without the need of a centralized authentication server, a hash-based token and identification system was introduced which additionally enables the id based logging of event message corresponding to one merchant or consumer. One of the causes to implement this decentralized authentication system was to reduce the amount of requests during the simulation. Described in the previous section, the gain of flexibility and scalability of services withing a microservice architecture goes along with the cost of an increased communication overhead. With this knowledge and experiences of a first prototype, the goal was to reduce the amount of request being sent over the network for streamlining a simulation and its necessary communication. 
In the following, we will briefly outline some of the main challenges which were solved.

%
\subsection{Decentralized Authorization}
% Owner: 
% Reviewed:
%

%
\subsection{Event log analysis with Kafka and Flink}
% Owner: 
% Reviewed:
%

%Besides the socket.io interface, the kafka-reverse-proxy also offers a RESTful API to fetch detailed logs for data-driven pricing algorithms. Based on  to learn their strategies appropriately. To ensure a fair starting point and competition, the single merchants may only access data out of their own scope which is enforced by the authentication token mechanisms. In this way, one merchant is not able to see the sales of another merchant.

%
\subsection{Challenges with high-density inter-service communication}
% Owner: 
% Reviewed:
%

%
\subsection{Simulation Limits and bottleneck evaluation}
% Owner: 
% Reviewed:
%

@TODO: how do the services interact, how do we secure some major challenges in short sentences.

No ticks or such, but completely free and dynamic, every merchant can check or update prices at any time -> close to real life (unlike eg http://www.informsrmp2017.com/description-challenge.pdf)

\begin{itemize}
\item event logs, kafka etc
\item fraud / cheating (and explain, why the consumer is required to send the price when buying products)
\item (inter service communication (via REST and connection pools))
\item (where are limits / bottlenecks?)
\end{itemize} 

@Todo: include data
Simulation 1 Stunde, 4000 Produkte à 4 Qualitäten, max 1000 Aktionen (Updates / Käufe) pro Minute, 750 Offer pro Merchant
06: 25 MB
05: 2,3 GB
04: 24 MB
03: 4 MB
02: 2 MB
01: 60 KB