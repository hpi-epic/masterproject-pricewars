%
\section{Evaluation}
\label{sec:evaluation}
% Owner: Jani
% Reviewed:
%
Within the first simulation rounds it turns out that an exemplary data-driven approach based on a logistic regression \citep{hosmer2013applied} pricing model promise higher profits. It is noteworthy, that this is achieved through better consumer behavior adjustments in pricing than common rule based pricing strategies do.  
Undoubtedly, the data-driven approach adapts quicker to changes in consumer behavior than conventional approaches.


\section{Future Work}
\label{sec:Future_Work}
% Owner: Jani
% Reviewed:
%
We are aware that the current solution is far from being perfect. One is already able to simulate different market situations based a variety of consumer behaviors and competing merchants, however, this solution can be extended to cover even more possibilities. Currently, the producer may provide goods without any expiration date. But when thinking about plane or festival tickets, perishables or any other short-life products, those can be also included in a later step as simulation content. If so, the producer may include an expiration date and optionally a cap of items (for the air plane case) which consequently has to be checked and verified by the marketplace. Additionally, the behaviors by the consumer as well as within the merchants may need to react one those additional attributes.\\

Another very interesting case to cover in the near future could be consumer ratings and how they influence pricing strategies. Having now such an environment to simulate different market situations and consumer demands, different consumer ratings may also influence pricing strategies significantly.

\section{Conclusion}
\label{sec:conclusion}
% Owner: Jani
% Reviewed:
%
In this work, we presented a distributed and scalable platform to imitate market situations and simulate dynamic pricing algorithms and their effects with potential real world settings. With this toolkit, handler may be enabled to test their pricing strategies appropriately before releases them into the real world where they can create huge loses. Simultaneously, students and researchers may now implement and test how pricing strategies react and interact based on influences and against each other.\\

The source code and its technical documentation will be publicly available at github.com \footnote{\url{https://github.com/hpi-epic/masterproject-pricewars}}
while a screencast is accessible under youtube\footnote{\url{https://www.youtube.com/watch?v=bqXSi5cv8cE}}.\\

\textbf{Acknowledgements}:
As part of this elaboration, special thanks goes to Dr. Rainer Schlosser and Martin Boissier for their continuous support and supervision during the workshop. Also, thank belongs to all members of the Enterprise Platform and Integration Concepts research group for their fruitful discussions during the intermediate and final presentation.