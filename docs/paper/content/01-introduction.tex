%
\section{Introduction}
% Owner: Jani
% Reviewed: 
%
While for the last couple of centuries prices were mainly defined through rule-based actions, more and more companies start basing their pricing calculations and strategies on technology and data-driven processes. Not only the actual computation of profit and cost is nowadays algorithm driven, but also its update and review strategy.
One of the most competitive and advanced fields is the algorithmic trading or high-frequency trading on stock exchanges. But also each one of us experiences nowadays technology driven price calculation on online marketplace like amazon, which we will hereinafter referrer to as dynamic pricing. 

Currently, pricing strategies and algorithms exist, but handlers of those mechanism lack of the possibility to test them appropriately before releasing them into the real world where they can create huge loses \citep{uflacker2016ertragsmanagement} \citep{schlosser2016optimal} \citep{schlosser2016stochastic} \citep{schlosser2016survive}. We picked up the challenge to create such an environment, imitating different market situations and therefore testing how pricing strategies react and interact based on influences and against each other.\\

\textbf{Contribution:} In this work, we briefly elaborate our process of building a distributed and scalable platform to imitate market situations and simulate dynamic pricing algorithms and their effects with potential real world settings.
Therefore, the following Chapter \ref{sec:Architecture} contains a short introduction into the underlying architecture of the platform.
The choreography of different services is described in \ref{sec:Choreography}. Chapter \ref{sec:Behaviors} will provide insights in the already implemented algorithms and their behaviors.
A user facing interface on top of the RESTful API used for the communication between the services is delineated in \ref{sec:ui} and finally Chapter \ref{sec:conclusion} concludes this elaboration. \\

% Owner: Johanna
\textbf{Design Goals:} While building this platform, we had a set of specific design goals and restrictions in mind, that we wanted to fulfill to make the platform and the resulting simulation as realistic, dynamic and reactive as possible: 
\begin{itemize}
    \item \textbf{Reactivity:} Allow pricing algorithms to do fully dynamic price updates and look-ups at any time without being refrained by discrete time intervals to enable truly reactive pricing strategies.
    \item \textbf{Security:} Prevent fraud in simulation setups with multiple participants to allow the simulation to be used for competitions or education purposes. Fraud could be the deliberate influence on the expenses or profits of other pricing algorithms or the usage of data (e.g. for learning purposes) that is normally not accessible.
    \item \textbf{Realism:} Enforce real-life restrictions present in every big real-life marketplaces such as limited requests per time interval to avoid constant polling of the market situation.  
    \item \textbf{Flexibility:} Offer the greatest possible flexibility in terms of scalability and adaptivity of the system to make the system easily expandable and adaptable to new possible design goals or user needs.
\end{itemize}

%Check intro in \url{http://faculty.chicagobooth.edu/workshops/omscience/pdf/Spring%202016/Popescu.pdf}

%
\section{Related Work}
% Owner: 
% Reviewed: 
%
(only own section if we find enough) \\

Check literature review in \url{http://faculty.chicagobooth.edu/workshops/omscience/pdf/Spring%202016/Popescu.pdf}