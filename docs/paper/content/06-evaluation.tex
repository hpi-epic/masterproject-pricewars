%
\section{Evaluation}
\label{sec:evaluation}
% Owner: Jani
% Reviewed:
%
We used the platform built to evaluate the performance of the already provided merchant behaviors. Thus, we wanted to examine the quality of and the possibilities offered by the simulation. We also evaluated the amount of data generated by the simulation and tried to identify bottlenecks for the system's performance.

%Within the first simulation rounds, we already gained several interesting insights regarding the merchant performance and the system load needed to be handled. Additionally, the generated data sizes will be addressed.

\subsection{Merchant Performance}
\label{sec:merchant_evaluation}
% Owner: Jani
% Reviewed:
%
We assumed that the exemplary data-driven approach based on a logistic regression pricing model promises higher profits than common rule-based behaviors because of its better adjustment to the consumer behavior in pricing. To test this hypothesize, we performed a one-on-one evaluation of the 6 provided merchant behaviors from \cref{sec:Behaviors_Merchants}. We ran each evaluation with the following settings:

\begin{itemize}
    \item Duration: 20 minutes
    \item Consumer:
    \begin{itemize}[nosep]
        \item Max. 20 purchases per minute
        \item 100\% logit-behavior
    \end{itemize}
    \item Producer: 1 product with 4 qualities
    \item Merchant: 
    \begin{itemize}
        \item 3 products in stock per merchant
        \item 2 updates per second
    \end{itemize}
\end{itemize}

The results can be found in \cref{table:merchant-evaluation-matrix}.

{
\setlength\extrarowheight{2pt}
\begin{table}[ht]
\centering
\caption{Performance results of the one-on-one merchant behavior evaluation. The letter in each cell denotes the merchant of the pair that gained the higher profit, i.e. won the simulation. The number below the letter tells how much more profit than their competitor they gained. A: Cheapest, B: Second Cheapest, C: Random Thee, D: Two-Bound, E: Fixed Price, F: Machine-Learning }
\label{table:merchant-evaluation-matrix}
\begin{tabular}{|r||c|c|c|c|c|c|c|}
\hline
 & \textbf{A} & \textbf{B} & \textbf{C} & \textbf{D} & \textbf{E} & \textbf{F}  \\ \hline \hline
\textbf{A} & & \cellcolor{lightgray} & \cellcolor{lightgray} & \cellcolor{lightgray} & \cellcolor{lightgray} & \cellcolor{lightgray} \\ \hline
\textbf{B} & \evalresult{B}{76,16} & & \cellcolor{lightgray} & \cellcolor{lightgray} & \cellcolor{lightgray} & \cellcolor{lightgray} \\ \hline
\textbf{C} & \evalresult{A{76,16} & & & \cellcolor{lightgray} & \cellcolor{lightgray}& \cellcolor{lightgray} \\ \hline
\textbf{D} & & & & & \cellcolor{lightgray} & \cellcolor{lightgray} \\ \hline
\textbf{E} & & & & & & \cellcolor{lightgray} \\ \hline
\textbf{F} & & & \evalresult{F}{70} & & &  \\ \hline
\end{tabular}
\end{table}
}

We also ran an evaluation with all merchants running at once to see whether the data-driven merchant would beat the other, rule-based merchants. The same settings as above were used except for the consumer who bought a maximum of 60 items per minute to account for the increased number of merchants. The results can be seen in \cref{table:all-merchants-evaluation}.

{
\setlength\extrarowheight{2pt}
\begin{table}[ht]
\centering
\caption{Performance results of the all merchant evaluation. The numbers are the profit gained by each merchant after 20 minutes of simulation, i.e. the higher the better. A: Cheapest, B: Second Cheapest, C: Random Thee, D: Two-Bound, E: Fixed Price, F: Machine-Learning}
\label{table:all-merchants-evaluation}
\begin{tabular}{|c|c|c|c|c|c|c|}
\hline
\textbf{A} & \textbf{B} & \textbf{C} & \textbf{D} & \textbf{E} & \textbf{F}  \\ \hline \hline
 & & & & &  \\ \hline
\end{tabular}
\end{table}
}

The results clearly show that the data-driven merchant outperforms all other, rule-based merchants in a one-on-one setup. This is in line with the hypothesize we formulated above. However, the simulation with six merchants at once disproves this hypothesize - the data-driven merchant only had the second-best result and was beaten by the two-bound, rule-based strategy, that clearly lost against the data-driven merchant in the one-on-one simulation. 
This result shows very well how important and unpredictable the influence of interaction effects between multiple merchants can be on the results and that simply testing and evaluating strategies on a one-on-one basis is not sufficient to assess its performance.

Thus, these results demonstrate the usefulness and possible applications of the platform built. It allows to account for these unforeseeable interaction effects by enabling the user to run quick and complex simulations of possible real-world setups with an almost arbitrarily high variety of merchants and strategies. At the same time it offers comprehensive evaluation metrics, giving the user immediate feedback on the performance of each merchant, fulfilling many of the design goals we postulated at the beginning in \cref{sec:Design_Goals}. 

Of course, further tests have to be run with other data-driven strategies as well, especially to investigate how data-driven strategies perform when they play against each other and not only against rule-based strategies. 


%
\subsection{Produced Data}
% Owner: Johanna
% Reviewed:
%

We did an evaluation of the data size produced by a simulation using the following benchmarks:
\begin{itemize}
    \item Duration: 1 hour
    \item Products: 4000 with 4 different qualities each, i.e. 16,000 products in total
    \item Merchants: 5 merchants
    \begin{itemize}[nosep]
        \item 750 products in stock per merchant
        \item 17 requests per second per merchant
    \end{itemize}
    \item Consumer: Max. 1000 purchases per minute
\end{itemize}

The amount of data generated and persisted per service can be seen in \cref{table:generated_data}.

%\begin{table}[ht]
%\centering
%\caption{Data size generated by a 1-hour simulation with x merchants, 4000 products and max. 2000 actions per minute}
%\label{table:generated_data}
%\begin{tabular}{|l|r|}
%\hline
%\textbf{Service} & \textbf{Data Size} \\ \hline \hline
%Kafka            & 2.3 GB             \\ \hline
%Merchants (total)& 25 MB              \\ \hline
%Marketplace      & 24 MB              \\ \hline
%Producer         & 4 MB               \\ \hline
%UI               & 2 MB               \\ \hline
%Consumer         & 60 KB              \\ \hline
%\end{tabular}
%\end{table}

\begin{table}[ht]
\centering
\caption{Data size generated by a 1-hour simulation with 5 merchants and 4000 products.}
\label{table:generated_data}
\begin{tabular}{|l|r|}
\hline
\textbf{Service} & \textbf{Data Size} \\ \hline \hline
Kafka            & 173 MB             \\ \hline
Merchants (total)& 109 MB             \\ \hline
Marketplace      & 72 MB              \\ \hline
Producer         & 13 MB              \\ \hline
UI               & 1.7 MB             \\ \hline
Consumer         & 800 KB             \\ \hline
\end{tabular}
\end{table}

\todo[inline]{is this really data generated in one hour? I think it also includes data that is just always present, eg the 4 MB at the producer won’t change anymore, they are just due to the high number of products } 
\todo[inline]{we started with 7 merchants, one stopped at the beginning and two died later - do we say we tested it with 4, 5, 6 or 7 merchants...?}

\subsection{System Load}
\label{sec:system_evaluation}
% Owner: 
% Reviewed:
%
The developed system is able to handle very high loads in terms of the number of merchants, offers or products. The only potentially problematic bottlenecks are the user interface (see also \cref{sec:FutureWork}) and the amount of produced data as stated in the previous section, depending on the duration of the simulation and the underlying machine's capacities. 

As shown during the evaluation of the data sizes, a very high number of products is no problem for the system - only the creation of those 4000 products took some time, but the overall performance of the simulation itself was not affected negatively. 

The load on the marketplace is determined by the number of requests per minute that a merchant and the consumer can make. This number is adjustable to allow the simulation of times of higher or lower demand on the consumer side (such as Christmas) or to simply speed up the simulation if all parameters are increased equally. As implied above, we ran simulations with more than 100 requests per second without any problems. Also, simulations with 6 merchants at once, doing 2 updates per second each, and a consumer buying 60 products per minute, were absolutely no problem for the system. 

%\todo[inline]{kann jemand abschätzen, was das maximum an gesamt-requests ist, das wir bisher problemlos getestet haben/handeln könnten?}
%\todo[inline]{können wir ne abschätzung geben, wie viele merchants max laufen können (gegeben max anzahl von requests von x/minute)?}


\section{Future Work}
\label{sec:FutureWork}
% Owner: Jani, Johanna
% Reviewed: 
%
While providing an easy and comprehensive way to simulate different market situations based on a variety of consumer behaviors and competing merchants, the current solution still has a lot of potential for extension and improvement. 

In the current setup, the producer only offers goods without any expiration date. But when thinking about plane or festival tickets, perishables or any other short-life products, pricing strategies might perform very differently and have to adapt to completely new features. Thus, it would be interesting to add the possibility to offer such perishable products. In consequence, a more comprehensive notion of time would have to be introduced throughout the whole system, and the marketplace would have to check and verify a product's expiration date with the producer. Additionally, the behaviors of the consumer as well as within the merchants need to react on those additional attributes.

Another very interesting case to cover in the near future could be consumer ratings and how they influence pricing strategies. Having now such an environment to simulate different market situations and consumer demands, different consumer ratings may also influence pricing strategies significantly. 

Lastly, it would be interesting to extend the simulation to support buying strategies within merchants in addition to pricing strategies, meaning that merchants no longer receive a fixed number of random products, but can decide on what, when and how much they want to buy themselves. This would add a lot of additional complexity to the design of merchant strategies, but would be an even closer simulation of the real world.

We are also aware that the current solution is far from being perfect. Components of the current solution that need further revision in the future are, in particular, the UI and also the Kafka-related components.

The user interfaces turned out to be a bottleneck in the current setup when the simulation is under a high load, i.e. there are many price updates and sales. Especially the real-time price interaction graphs start to lag heavily or even crash the browser under high load, making parts of the UI almost unusable. To solve this problem, a complete refactoring of this UI component or a switch of the underlying third party charts library might be necessary.

The Kafka-components might need to be refactored in terms of the size of the produced data. As shown in \cref{sec:evaluation}, the amount of data produced in just one hour of simulation is above 2 GB. While that amount should be rather easy to handle on most systems, potential long term simulations - as might be needed for more sophisticated data-driven pricing strategies or long-term algorithm evaluations - might produce more problematic data amounts. A more considerate handling of outdated data as well as a stronger compression of current data might be needed. 


%OLD: We are aware that the current solution is far from being perfect. One is already able to simulate different market situations based on a variety of consumer behaviors and competing merchants, however, this solution can be extended to cover even more possibilities. Currently, the producer may provide goods without any expiration date. But when thinking about plane or festival tickets, perishables or any other short-life products, those can be also included in a later step as simulation content. If so, the producer may include an expiration date and optionally a cap of items (for the air plane case) which consequently has to be checked and verified by the marketplace. Additionally, the behaviors by the consumer as well as within the merchants may need to react one those additional attributes.\\
